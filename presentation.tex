\documentclass[aspectratio=1610]{beamer}
\usetheme{KTH}

% remove this if using XeLaTeX or LuaLaTeX
\usepackage[utf8]{inputenc}
\usepackage{graphics}
\usepackage{graphicx}
\usepackage{booktabs}
\usepackage{ragged2e}
\usepackage{lipsum}
\usepackage{minted}
\usepackage{tikz}
\usepackage{array}
\usepackage{algorithm,algorithmicx}
\usepackage{algpseudocode}
\usepackage{amsmath,amsfonts,amssymb}
\usepackage[export]{adjustbox}
\newcommand*{\field}[1]{\mathbb{#1}}
\newtheorem*{teo}{Teorema}
\newtheorem*{defn}{Definición}
\newtheorem*{lem}{Lema} 
\begin{document}

%------------------------------------------------
\begin{frame}[noframenumbering,plain]

  \vspace{0.02\textheight}
  
\begin{columns}[]
\column{37em}
\Large{\centerline{\usebeamercolor[fg]{title}Una interpretación matemática de la Ecuación de Schrödinger}}

\vspace{0.1\textheight}

\vspace{0.1\textheight}

\small{\centerline{Armando Miguel Trejo Marrufo}}
\scriptsize{\centerline{\tt Asesor: Pablo Castañeda}}
\scriptsize{\centerline{}}
\end{columns}
\end{frame}

%------------------------------------------------
\usebackgroundtemplate{\vbox{\null\vspace{3mm}
  \hspace{3mm}\pgfuseimage{kthlogosmall}\par
  \vspace{72mm}\hbox{\hspace{-75mm}\pgfuseimage{kthplatta}}}}

%------------------------------------------------
\begin{frame}
\begin{columns}
\column{37em}
\vspace{1cm}
\Huge{\centerline{\usebeamercolor[fg]{title}Ecuación Dependiente del Tiempo}}
\end{columns}
\end{frame}


%------------------------------------------------
\begin{frame}
\frametitle{Ecuación Dependiente del Tiempo de Schrödinger}

Dado que 
\begin{align*}
    -\frac{\hslash^2}{2m}\frac{\partial^2\psi(x,t)}{\partial x^2} + \mathcal{V}(x,t)\psi(x,t) = i\hslash\frac{\partial\psi(x,t)}{\partial t},
\end{align*}
buscamos modelar el comportamiento de una partícula libre, es decir, $\mathcal{V}(x,t) = 0$. Así,
el problema de valor inicial (PVI) a resolver es:
    \begin{align}
         \left\{ \begin{array}{ll}
         \dfrac{\partial\psi}{\partial t} = \dfrac{i\hslash}{2m}\dfrac{\partial^2\psi}{\partial x^2},\:\:\:x\in\field{R},\:t>0 \\
         \psi(x,0) = \psi_{0}(x),\:x\in\field{R} \\
         \end{array}
\right.
    \label{eq:TISchrodinger}
    \end{align}
donde pediremos que $\psi_{0} \in L^2(\field{R},\field{C})$.
\begin{columns}
\column{37em}
\end{columns}
\end{frame}



%------------------------------------------------
\begin{frame}{Solución general de la Ecuación Dependiente del Tiempo de Schrödinger}
   
\begin{teo}
    Supongamos $\psi_{0} \in L^2(\field{R},\field{C})$. Definamos $\psi(x,t)$ por:
    \begin{align*}
        \psi(x,t) = \sqrt{\frac{m}{2\pi it\hslash}}\int_{-\infty}^{\infty}\exp\bigg[-\frac{m}{2it\hslash}(x-y)^2\bigg]\psi_{0}(y) \: dy.
    \end{align*}
    Entonces $\psi(x,t)$ satisface el PVI $\eqref{eq:TISchrodinger}$.
    \label{teo:Rev4}
\end{teo}
    
\begin{columns}
\column{37em}
\end{columns}
\end{frame}

%------------------------------------------------
\begin{frame}{Solución Débil}
   
Qué es una solución débil?   
   
\begin{lem}
    Sea $\psi(x,t)$ una función suave, en el sentido de tener primera derivada espacial y temporal, entonces $\psi(x,t)$ satisface la ecuación de Schrödinger en el sentido clásico si y solo si $\psi(x,t)$ satisface la ecuación de Schrödinger en el sentido débil.
\end{lem}
    
\begin{columns}
\column{37em}
\end{columns}
\end{frame}


%------------------------------------------------
\begin{frame}{Qué es un Paquete de Ondas?}
   
Dentro de un paquete de ondas, sus componentes se mueven a distintas velocidades. Es útil pensar que éstas varían en el tiempo y en el espacio.  
\begin{figure}[h]
    \centering
    \includegraphics[width = 9cm, height = 3cm]{figs/GroupandPhase.png}
\end{figure}
Por un lado, la velocidad a la cuál la parte externa (la cobertura) se propaga en el espacio a velocidad se conoce como velocidad de grupo y queda determinada por $A(x,t)$. Por otro lado, la rápidez a la cual las ondas individuales dentro de la cobertura se mueven se conoce como velocidad fase y queda determinada por $\theta(x,t)$. 
    
\begin{columns}
\column{37em}
\end{columns}
\end{frame}

%------------------------------------------------
\begin{frame}{Paquete de ondas como Condición Inicial}
   
Al suponer que la fase de onda inicial es $\theta(x,0) = (p_{0}x) / \hslash$, debido a que el momento inicial de la partícula es $p_{0}$, tendremos el Problema de Valor Inicial siguiente: 
\begin{align}
         \left\{ \begin{array}{ll}
         \dfrac{\partial\psi}{\partial t} = \dfrac{i\hslash}{2m}\dfrac{\partial^2\psi}{\partial x^2} , \:x\in\field{R}, \:t>0,\\
         \psi(x,0) = \psi_{0}(x) = A_{0}(x)e^{ip_{0}x/h}, \:x\in\field{R}. \\
         \end{array} \label{eq:Paqonda}
\right.
    \end{align}
donde notamos que $|\psi_{0}(x)|\leq A_{0}(x)\:\:\: \forall x\in\field{R}$, es decir, $A_{0}(x)$ envuelve al paquete de ondas.
    
\begin{columns}
\column{37em}
\end{columns}
\end{frame}

%------------------------------------------------
\begin{frame}{Solución del Paquete de ondas como Condición Inicial}
   
\begin{teo}
    La solución del PVI $\eqref{eq:Paqonda}$ con condición inicial $\theta(x,0) = p_{0}x/\hslash$ y $A(x,0) = A_{0}(x)$ está dado por:
    \begin{align*}
        \theta(x,t) & = \frac{p_{0}}{\hslash}\bigg(x - \frac{p_{0}}{2m}t\bigg),
        \\
        A(x,t) & = A_{0}\bigg(x-\frac{p_{0}}{m}t\bigg),
    \end{align*}
    las cuáles dan como solución a la Ecuación Libre de Schrödinger el paquete de ondas:
    \begin{align}
        \psi(x,t) = A_{0}\bigg(x-\frac{p_{0}}{m}t\bigg)\exp\bigg[i\frac{p_{0}}{\hslash}(x - \frac{p_{0}}{2m}t)\bigg].
        \label{eq:Apoyo7}
    \end{align}
\end{teo}
    
\begin{columns}
\column{37em}
\end{columns}
\end{frame}

%------------------------------------------------
\begin{frame}
\begin{columns}
\column{37em}
\vspace{1cm}
\Huge{\centerline{\usebeamercolor[fg]{title}Ecuación Independiente del Tiempo}}
\end{columns}
\end{frame}


%------------------------------------------------
\begin{frame}{}
La Ecuación Independiente del Tiempo de Schrödinger tiene la siguiente forma:   
    \begin{align}
        \frac{-\hslash^2}{2m}\Psi''(x) + \mathcal{V}(x)\Psi(x) = E \Psi(x)
        \label{eq:EITS}
    \end{align}

\begin{columns}
\column{37em}
\end{columns}
\end{frame}

%------------------------------------------------
\begin{frame}{Pozo Infinito}
   
Consideramos el caso de confinar a una partícula en un intervalo finito $[0 , A]$ con $A$ un valor fijo, para esto, pedimos que la energía potencial en este intervalo sea nula.
\begin{align*}
        \mathcal{V}(x) = 
        \left\{ \begin{array}{ll}
        0, \:\:\:  x \in [0,A],
        \\
        \kappa, \:\:\: x \notin [0,A],
        \end{array}
        \right.
\end{align*}
donde $\kappa$ toma valores muy grandes, es decir, $\kappa \to\infty$. Además, pedimos $\Psi(x) = 0$ en las fronteras, para que la partícula no escape de la región. Así, tendremos 
\begin{align*}
    \frac{d^2\Psi}{dx^2} = -k^2\Psi(x),
\end{align*}
con $k^2 = (2mE)/\hslash^2$.
    
\begin{columns}
\column{37em}
\end{columns}
\end{frame}


%------------------------------------------------
\begin{frame}{Pozo Finito}
   
Al tomar una región par $[-A , A]$ para obtener soluciones pares e impares y considerar que la energía potencial es de la siguiente forma  
\begin{align}
        \mathcal{V}(x) = 
        \left\{ \begin{array}{ll}
        -C, \:\:\:  x \in [-A,A],
        \\
        0, \:\:\: x \notin [-A,A],
        \end{array}
        \right.
        \label{eq:PotencialFinito}
\end{align}
donde $C$ es positiva. Así, cuando $E < 0$ tendremos que 
\begin{align}
        \frac{d^2\Psi}{dx^2}= 
        \left\{ \begin{array}{ll}
        -(c-\mathcal{E})\Psi,& \:\:\:  |x| \leq A,
        \\
        \mathcal{E}\Psi,& \:\:\: |x| > A.
        \end{array}
        \right.
        \label{eq:SCsimple}
\end{align}
donde consideramos los siguientes términos $\mathcal{E} = (-2mE)/\hslash^2$ y \\ $c = (2mC)/\hslash^2$
    
\begin{columns}
\column{37em}
\end{columns}
\end{frame}



%------------------------------------------------
\begin{frame}{Solución Par}
   
    
\begin{columns}
\column{37em}
\end{columns}
\end{frame}


%------------------------------------------------
\begin{frame}{Solución Impar}
   
    
\begin{columns}
\column{37em}
\end{columns}
\end{frame}


%------------------------------------------------
\begin{frame}{Tunelaje Cuántico}
   
En este caso $E > 0$, así $\mathcal{E} = 2mE/\hslash^2$, la ecuación de interés es:
\begin{align}
        \frac{d^2\Psi}{dx^2}= 
        \left\{ \begin{array}{ll}
        -\mathcal{E}\Psi,& \:\:\:  x \in (-\infty,-A)\cup (A,\infty),
        \\
        -(c+\mathcal{E})\Psi& \:\:\: |x| \leq A.
        \end{array}
        \right.
        \label{eq:SCsimple}
\end{align}
con correspondiente solución:
\begin{align}
        \Psi(x) = 
        \left\{ \begin{array}{ll}
        Me^{i\sqrt{\mathcal{E}}x}+Be^{-i\sqrt{\mathcal{E}}x}, & x<-A,
        \\
        T\sen(qx) + D\cos(qx), & x \in [-A,A],
        \\ Fe^{i\sqrt{\mathcal{E}}x},& x > A.
        \end{array}
        \right.
        \label{eq:Tunelaje}
\end{align}
donde al buscar que $\Psi(x)$ y $\Psi'(x)$ sean continuas en los límites de la región $\pm A$, obtendremos un sistema de ecuaciones para determinar las constantes $M, B, T, D$ y $F$. 

    
\begin{columns}
\column{37em}
\end{columns}
\end{frame}
%------------------------------------------------
\begin{frame}{Tunelaje Cuántico}
   
Al resolver el sistema de ecuaciones anteriores, obtenemos el valor del coeficiente de Reflexión $\mathcal{R}$
\begin{align*}
    \mathcal{R}(E) = \bigg[\frac{C^2}{4(C+E)E}\bigg]\sen^2(2qA)\frac{|F|^2}{|M|^2}.
\end{align*}
y del coeficiente de Transmisión $\mathcal{T}$
\begin{align*}
    \mathcal{T}(E) = 
    \bigg[1+\bigg(\frac{C^2}{4(C+E)E}\bigg)\sen^2(2qA)\bigg] ^{-1}.
\end{align*}

    
\begin{columns}
\column{37em}
\end{columns}
\end{frame}


%------------------------------------------------
%\begin{frame}
%\frametitle{Collaborative }
%\begin{columns}
%\column{37em}
%\begin{itemize}\itemsep1em
%  \justifying
%  \item A \textcolor{Ocean}{shared platform} for all %\textcolor{TextGreen}{data science} contributors
%  \item Open data science \textcolor{Ocean}{tools at scale}
%  \item \textcolor{TextGreen}{Self-service} access to %\textcolor{Ocean}{data}, \textcolor{Ocean}{storage}, and %\textcolor{Ocean}{compute}
%  \item A complete \textcolor{darkred}{pipeline} from data to deployment
%  \item Collaborative \textcolor{Ocean}{development tools}
%  \item \textcolor{Ocean}{Management} and \textcolor{Ocean}{reproducibility}
%\end{itemize}
%\end{columns}
%\end{frame}

%------------------------------------------------
\begin{frame}
\begin{columns}
\column{37em}
\vspace{1cm}
\Huge{\centerline{\usebeamercolor[fg]{title}Thanks!}}
\end{columns}
\end{frame}

\end{document}
