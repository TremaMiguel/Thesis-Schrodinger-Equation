\documentclass[aspectratio=1610]{beamer}
\usetheme{KTH}
\usepackage[utf8]{inputenc}
\usepackage{graphics}
\usepackage{graphicx}
\usepackage{booktabs}
\usepackage{ragged2e}
\usepackage{lipsum}
\usepackage{minted}
\usepackage{tikz}
\usepackage{array}
\usepackage{algorithm,algorithmicx}
\usepackage{algpseudocode}
\usepackage{amsmath,amsfonts,amssymb}
\usepackage[export]{adjustbox}
\newcommand*{\field}[1]{\mathbb{#1}}
\newtheorem*{teo}{Teorema}
\newtheorem*{defn}{Definición}
\newtheorem*{prop}{Proposición}
\newtheorem*{lem}{Lema} 
\newtheorem*{demo}{Demostración}
\begin{document}

%------------------------------------------------
\begin{frame}[noframenumbering,plain]

\vspace{0.02\textheight}
  
\begin{columns}[]
\column{37em}
\Large{\centerline{\usebeamercolor[fg]{title}Una interpretación matemática de la Ecuación de Schrödinger}}

\vspace{0.1\textheight}

\vspace{0.1\textheight}

\small{\centerline{Armando Miguel Trejo Marrufo}}
%\scriptsize{\centerline{\tt Asesor: Pablo Castañeda}}
\scriptsize{\centerline{}}
\end{columns}
\end{frame}

%------------------------------------------------
\usebackgroundtemplate{\vbox{\null\vspace{3mm}
  \hspace{3mm}\pgfuseimage{kthlogosmall}\par
  \vspace{72mm}\hbox{\hspace{-75mm}\pgfuseimage{kthplatta}}}}

%------------------------------------------------
\begin{frame}
\frametitle{}
\begin{columns}
\column{37em}
\begin{itemize}\itemsep1em
  \justifying
  \item \textcolor{Ocean}{¿Antecedentes?} 
  \textcolor{TextGreen}{ Dualidad onda-partícula} $\implies$ Louis de Broglie y Albert Einstein.
  
  \begin{figure}%
    \centering
    {{\includegraphics[width=4.5cm, height = 3cm]{figs/DeBroglie.jpg}}}%
    \qquad
    {{\includegraphics[width=4.5cm, height = 3cm]{figs/Einstein.jpg}}}%
    
    \label{fig:example}%
\end{figure}
  
  \item  \textcolor{Ocean}{¿Cual es su importancia?} \textcolor{TextGreen}{ Dinámica} $\implies$ describir la evolución de la función de onda en el tiempo.
  
  \begin{figure}%
    \centering
    {{\includegraphics[width=4.5cm, height = 3cm]{figs/Schrodinger.jpg}}}%
    \label{fig:example}%
\end{figure}
\end{itemize}
\end{columns}
\end{frame}








%------------------------------------------------
\begin{frame}
\begin{columns}
\column{37em}
\vspace{1cm}
\Huge{\centerline{\usebeamercolor[fg]{title}Ecuación Dependiente del Tiempo}}
\end{columns}
\end{frame}


%------------------------------------------------
\begin{frame}
\frametitle{ }
\begin{columns}
\column{37em}
\begin{itemize}\itemsep1em
  \justifying
  \item  \textcolor{Ocean}{La dependencia del tiempo} 
  \item  \textcolor{Ocean}{El Problema de Valores Iniciales} 
  \item  \textcolor{Ocean}{Soluciones Fuertes y Débiles} 
  \item  \textcolor{TextGreen}{Funciones de Prueba}
  \item  \textcolor{TextGreen}{El paquete de Ondas}
  \item  \textcolor{TextGreen}{La condición de variación lenta (SVEA)}
\end{itemize}
\end{columns}
\end{frame}




%------------------------------------------------
\begin{frame}
\frametitle{Ecuación Dependiente del Tiempo de Schrödinger}

La ecuación de Schrödinger representa el movimiento de la partícula en el tiempo:
\begin{align}
    -\frac{\hslash^2}{2m}\frac{\partial^2\psi(x,t)}{\partial x^2} + \mathcal{V}(x,t)\psi(x,t) = i\hslash\frac{\partial\psi(x,t)}{\partial t}.
    \label{eq:ECDepTiemp}
\end{align}
Para modelar una \textbf{partícula líbre} necesitamos 
\begin{align*}
    \mathcal{V}(x,t) = 0
\end{align*}  
y la condición inicial
\begin{align*}
    \psi(x,0) = \psi_{0}(x),
\end{align*}
donde $\psi_{0}(x) \in L^{2}(\field{R},\field{C})$

\begin{columns}
\column{37em}
\end{columns}
\end{frame}

%--------------------------------------
\begin{frame}

El problema de valor inicial (PVI) a resolver es:
    \begin{align}
         \left\{ \begin{array}{ll}
         \dfrac{\partial\psi}{\partial t} = \dfrac{i\hslash}{2m}\dfrac{\partial^2\psi}{\partial x^2},\:\:\:x\in\field{R},\:t>0 \\
         \psi(x,0) = \psi_{0}(x),\:x\in\field{R} \\
         \end{array}
\right.
    \label{eq:TISchrodinger}
    \end{align}
 
\vspace{0.02\textheight} 
    
\begin{teo}
    Supongamos $\psi_{0} \in L^2(\field{R},\field{C})$. Definamos $\psi(x,t)$ por:
    \begin{align*}
        \psi(x,t) = \sqrt{\frac{m}{2\pi it\hslash}}\int_{-\infty}^{\infty}\exp\bigg[-\frac{m}{2it\hslash}(x-y)^2\bigg]\psi_{0}(y) \: dy.
    \end{align*}
    Entonces $\psi(x,t)$ satisface el PVI $\eqref{eq:TISchrodinger}$.
    \label{teo:Rev4}
\end{teo}

\begin{columns}
\column{37em}
\end{columns}
\end{frame}
%------------------------------------------------
\begin{frame}{}

\begin{itemize}\itemsep1em
  \justifying
  \item  \textcolor{Ocean}{¿$\psi(x,t)$ es diferenciable en las variables $x,t$? $\implies$ \textbf{funciones de prueba}} 
\end{itemize}

\vspace{0.1\textheight} 

\begin{defn}
        Una función $\psi(x,t)$ satisface la ecuación libre de Schrödinger en el sentido débil, si para toda función de prueba $\chi$ se cumple:
        \begin{align} \int_{\field{R}^+}\int_{\field{R}}\psi(x,t)\bigg[\frac{\partial\chi}{\partial t}-\frac{i\hslash}{2m}\frac{\partial ^2\chi}{\partial x^2}\bigg]\:dxdt = -\int_{\field{R}}\chi(x,0)\psi(x,0). \label{eq:Soldebil}
        \end{align}
        \label{lem:Rev5}
\end{defn}
    
\begin{columns}
\column{37em}
\end{columns}
\end{frame}

%------------------------------------------------
\begin{frame}{Soluciones Débiles de la Ecuación de Schrödinger}
   
\begin{lem}
    Sea $\psi(x,t)$ una función suave, entonces $\psi(x,t)$ satisface la ecuación de Schrödinger en el sentido clásico si y solo si $\psi(x,t)$ satisface la ecuación de Schrödinger en el sentido débil.
\end{lem}   
    
\begin{columns}
\column{37em}
\end{columns}
\end{frame}


%------------------------------------------------
\begin{frame}{¿Qué es un Paquete de Ondas?}
   
¿Las componentes de un paquete de ondas varían en el tiempo y en el espacio?
\begin{figure}[h]
    \centering
    \includegraphics[width = 9cm, height = 3cm]{figs/GroupandPhase.png}
\end{figure}
\begin{itemize}\itemsep1em
    \item \textcolor{}{Velocidad de Grupo $\implies$ $A(x,t)$}
    \item \textcolor{}{Velocidad de Fase $\implies$ $\theta(x,t)$}
\end{itemize}
    
\begin{columns}
\column{37em}
\end{columns}
\end{frame}

%------------------------------------------------
\begin{frame}{Paquete de ondas como Condición Inicial}
   
\begin{itemize}\itemsep1em
    \item \textcolor{}{La fase de onda inicial es $\theta(x,0) = (p_{0}x) / \hslash$}
    \item \textcolor{}{La amplitud inicial es $A_{0}(x)$}
\end{itemize} 

\vspace{0.1\textheight} 

El PVI que resolvemos es
\begin{align}
         \left\{ \begin{array}{ll}
         \dfrac{\partial\psi}{\partial t} = \dfrac{i\hslash}{2m}\dfrac{\partial^2\psi}{\partial x^2} , \:x\in\field{R}, \:t>0,\\
         \psi(x,0) = \psi_{0}(x) = A_{0}(x)e^{ip_{0}x/h}, \:x\in\field{R}. \\
         \end{array} \label{eq:Paqonda}
\right.
    \end{align}
donde $A_{0}(x)$ envuelve al paquete de ondas.
    
\begin{columns}
\column{37em}
\end{columns}
\end{frame}


%------------------------------------------------
\begin{frame}{Solución del Paquete de ondas como Condición Inicial}
   
\begin{teo}
    La solución del PVI $\eqref{eq:Paqonda}$ con condición inicial $\theta(x,0) = p_{0}x/\hslash$ y $A(x,0) = A_{0}(x)$ está dado por:
    \begin{align*}
        \theta(x,t) & = \frac{p_{0}}{\hslash}\bigg(x - \frac{p_{0}}{2m}t\bigg),
        \\
        A(x,t) & = A_{0}\bigg(x-\frac{p_{0}}{m}t\bigg),
    \end{align*}
    las cuáles dan como solución a la Ecuación Libre de Schrödinger el paquete de ondas:
    \begin{align}
        \psi(x,t) = A_{0}\bigg(x-\frac{p_{0}}{m}t\bigg)\exp\bigg[i\frac{p_{0}}{\hslash}(x - \frac{p_{0}}{2m}t)\bigg].
        \label{eq:Apoyo7}
    \end{align}
\end{teo}
    
\begin{columns}
\column{37em}
\end{columns}
\end{frame}

%---------------------------------------
\begin{frame}{Slowly Varying Envelope Approximation}

\begin{itemize}
    \item El paquete de ondas $A_{o}(x)$ tiene crestas chatas $\implies$ constante sobre los periodos de la función $e^{ip_{0}x/\hslash}$
    \vspace{0.1\textheight} 
    \item En notación Matemática
    \begin{align*} 
    A_{0}''(\kappa) \ll A_{0}(\kappa)\bigg(\frac{p_{0}}{\hslash}\bigg)^2,
    \end{align*}
\end{itemize} 
\end{frame}


%------------------------------------------------
\begin{frame}
\begin{columns}
\column{37em}
\vspace{1cm}
\Huge{\centerline{\usebeamercolor[fg]{title}Ecuación Independiente del Tiempo}}
\end{columns}
\end{frame}

%------------------------------------------------
\begin{frame}
\frametitle{ }
\begin{columns}
\column{37em}
\begin{itemize}\itemsep1em
  \justifying
  \item  \textcolor{Ocean}{Derivación a partir del caso dependiente del tiempo} 
  \item  \textcolor{Ocean}{Pozo Infinito} 
  \item  \textcolor{Ocean}{Pozo Finito} 
  \item  \textcolor{Ocean}{Tunelaje Cuántico}
\end{itemize}
\end{columns}
\end{frame}





%------------------------------------------------
\bgroup
\begin{frame}{}
Recordamos la Ecuación de Schrödinger
\begin{align}
    -\frac{\hslash^2}{2m}\frac{\partial^2\psi(x,t)}{\partial x^2} + \mathcal{V}(x,t)\psi(x,t) = i\hslash\frac{\partial\psi(x,t)}{\partial t}.
    \label{eq:ECDepTiemp}
\end{align}
\vspace{0.05\textheight} 
¿Cómo obtener el caso dependiente del tiempo? $\implies$ \textbf{separación de variables}
    \begin{align*}
    \psi(x,t) = \Psi(x)T(t).
    \end{align*}
\vspace{0.05\textheight} 
Así, la ecuación dependiente del tiempo es 
\begin{align}
\dfrac{\hslash^2}{2m}\dfrac{d^2}{dx^2}\Psi(x) + \mathcal{V}(x)\Psi(x) = E \Psi(x),
\label{eq:HamiltonianEigenValues}
\end{align}
donde $E \in \field{R}$

\end{frame}



%------------------------------------------------
\begin{frame}{Pozo Infinito}
 
¿Qué sucede cuando confinamos a una partícula a una región finita? 
\vspace{0.05\textheight} 
\textbf{Requerimos}
\begin{itemize}
    \item Energía potencial muy grande
    \begin{align*}
        \mathcal{V}(x) = 
        \left\{ \begin{array}{ll}
        0, \:\:\:  x \in [0,A],
        \\
        \kappa, \:\:\: x \notin [0,A],
        \end{array}
        \right.
    \end{align*}
    donde $A$ es un valor fijo.
    \item La partícula no escape de la región, es decir, $\Psi(x) = 0$.
    \vspace{0.05\textheight} 
    \item Modelar el comportamiento a través de
    \begin{align}
    \frac{d^2\Psi}{dx^2} = -(2mE)/\hslash^2\Psi(x),
    \label{eq:PozoInfinito}
    \end{align}
\end{itemize}
\begin{columns}
\column{37em}
\end{columns}
\end{frame}


%------------------------------------------------
\begin{frame}{}
\begin{columns}
\column{37em}
\end{columns}

 ¿Cuál es la solución general? $\implies$
 combinación lineal de senos y cosenos
\begin{align*}
    \Psi(x) = a\:\text{sen}(kx) + b\cos(kx),
\end{align*}   
donde encontrar los modos $k_{n}$ asociados a la partícula nos permite encontrar la energía $E_{n}$ asociada
\begin{align*}
    k_{n}=\frac{n\pi}{A}, E_{n} = \frac{\hslash^2n^2\pi^2}{2mA^2}, n \in \field{N}.
\end{align*}
con lo cual la solución dependiente de la posición es
\begin{align}
    \Psi_{n}(x) = \sqrt{\frac{2}{A}}\:\text{sen} \bigg(\frac{n\pi x}{A}\bigg), n \in \field{N},
    \label{eq:PsiPI}
\end{align}
    
\end{frame}


%------------------------------------------------
\begin{frame}{Pozo Finito}
  
¿Qué sucede si la energía potencial es finita?, es decir 
\begin{align}
        \mathcal{V}(x) = 
        \left\{ \begin{array}{ll}
        -C, \:\:\:  x \in [-A,A],
        \\
        0, \:\:\: x \notin [-A,A],
        \end{array}
        \right.
        \label{eq:PotencialFinito}
\end{align}
con $C > 0$. $\implies$ Obtendremos \textbf{soluciones pares e impares} 

\begin{equation*}
\Psi(x) =  \begin{cases}
             ae^{\sqrt{\mathcal{E}}x},& \:\:\:  x < -A, \\
             b\cos(qx), & \:\:\: |x| \leq A, \\
             ae^{-\sqrt{\mathcal{E}}x},& \:\:\: x > A,
       \end{cases} \quad
\Psi(x) =  \begin{cases}
        -ae^{\sqrt{\mathcal{E}}x},& \:\:\:  x < -A, \\
        \Tilde{a}\:\text{sen}(qx), & \:\:\: |x| \leq A, \\
        ae^{-\sqrt{\mathcal{E}}x},& \:\:\: x > A.
       \end{cases}
\end{equation*}
    
\begin{columns}
\column{37em}
\end{columns}
\end{frame}

%------------------------------------------------
\begin{frame}{}
\begin{figure}[h]
    \centering
    \includegraphics[width = 12cm, height = 6cm]{figs/Solimpar.png}
\end{figure}
 
\begin{columns}
\column{37em}
\end{columns}
\end{frame}

%------------------------------------------------
\begin{frame}{Tunelaje Cuántico}

\begin{figure}[h]
    \centering
    \includegraphics[width = 10cm, height = 4.5cm]{figs/Tunelaje.png}
\end{figure}
\textbf{\textcolor{Ocean}{¿Cómo se determinan estos eventos?}}
\begin{align*}
    \mathcal{R}(E) = \bigg[\frac{C^2}{4(C+E)E}\bigg]\sen^2(2qA)\frac{|F|^2}{|M|^2}, \:\:\:
    \mathcal{T}(E) = 
    \bigg[1+\bigg(\frac{C^2}{4(C+E)E}\bigg)\sen^2(2qA)\bigg] ^{-1}.
\end{align*}

    
\begin{columns}
\column{37em}
\end{columns}
\end{frame}
%------------------------------------------------
\begin{frame}
\begin{columns}
\column{37em}
\vspace{1cm}
\Huge{\centerline{\usebeamercolor[fg]{title}El Operador de Schrödinger}}
\end{columns}
\end{frame}


%------------------------------------------------
\begin{frame}
\frametitle{ }
\begin{columns}
\column{37em}
\begin{itemize}\itemsep1em
  \justifying
  \item  \textcolor{Ocean}{Operador Lineal y Autoadjunto} 
  \item  \textcolor{Ocean}{Dominio de la suma de Operadores} 
  \item  \textcolor{TextGreen}{Operador Laplaciano y Potencial}
  \item  \textcolor{TextGreen}{Teorema de Kato-Rellich}
  \item  \textcolor{TextGreen}{Generalizaciones de la Función Potencial $\mathcal{V}(x)$}
\end{itemize}
\end{columns}
\end{frame}



%------------------------------------------------
\begin{frame}{¿Qué es un Operador?}

Es un mapeo entre dos espacios de funciones
\vspace{0.05\textheight} 

\textbf{\textcolor{Ocean}{¿Qué propiedades buscamos?}}

\vspace{0.05\textheight} 
\begin{itemize}
    \item Lineal $\implies$ $A(\alpha\phi + \beta\Psi) = \alpha A\phi + \beta A \Psi$
    \vspace{0.03\textheight} 
    \item Acotado $\implies$ $||T\Psi||
    \leq C||\Psi||$
    \vspace{0.03\textheight} 
    \item Autoadjunto $\implies$ $A^{*}\phi = A\phi$ para toda $\phi \in Dom(A)$.
\end{itemize}
   
\begin{columns}
\column{37em}
\end{columns}
\end{frame}

%------------------------------------------------
\begin{frame}{}

\begin{defn}[Dominio de la suma de Operadores]
    Sean A,B dos operadores lineales no-acotados en $\mathcal{H}$, entonces
    \begin{align}
        Dom(A+B) := Dom(A)\cap Dom(B)
        \label{eq:SumadeOperadores}
    \end{align}
    donde $(A+B)\Psi = A\Psi + B\Psi$.
    \label{def:DomOPADJS}
\end{defn}
\begin{defn}[Espectro Puntual]
    Sea $A$ un operador lineal no-acotado en $\mathcal{H}$ y $\lambda\in\field{C}$. Si se cumple que 
    \begin{align*}
        (A-\lambda I)\phi = 0,
    \end{align*}
    para $\phi\in\mathcal{H}$ ($\phi\neq0$) entonces $\lambda$ está en el \textbf{espectro puntual}. 
\end{defn}

\begin{columns}
\column{37em}
\end{columns}
\end{frame}



%------------------------------------------------
\begin{frame}{El Operador de Schrödinger $\boldsymbol{-\Delta + V}$}

\begin{itemize}
    \item Sea el operador Laplaciano $-\Delta\Psi = \mathcal{F}^{-1}(|k|^{2} \hat{\Psi})$ con correspondiente dominio 
    \begin{align*} Dom(\Delta) = \{\Psi\in L^{2}(\field{R},\field{C}) \:|\:
       \: |k|^{2}\hat{\Psi}(k) \in L^{2}(\field{R},\field{C})\},
    \end{align*}
    \item Sea el operador Potencial $V$ con correspondiente dominio
    \begin{align*}  Dom(V) = \{
        \Psi\in L^{2}(\field{R},\field{C})\:|\:\mathcal{V}(x)\Psi(x)\in L^{2}(\field{R},\field{C})
        \}
    \end{align*}
    
\end{itemize}
    
\end{frame}

%------------------------------------------------
\begin{frame}{¿Bajo qué condiciones garantizo que la suma de dos operadores no-acotados sea auto-adjunta?}

\vspace{0.03\textheight} 

\begin{teo}[Kato-Rellich]
    Sean  A y B operadores lineales auto-adjuntos no-acotados en $\mathcal{H}$. Supongamos que $Dom(A)\subset Dom(B)$ tal que
    \begin{align*}
        ||B\Psi|| \leq a||A\Psi|| + b||\Psi||
    \end{align*}
    entonces A+B es auto-adjunto en el Dom(A).
    \label{teo:K-R}
\end{teo}    
\end{frame}


%------------------------------------------------
\begin{frame}{¿El operador de Schrödinger es autoadjunto?}

¿Qué sucede cuando $\mathcal{V}(x)$ no es una función constante? \\
\vspace{0.04\textheight}
Por ejemplo, una función definida a trozos

\begin{align*}
        \mathcal{V}(x) = 
        \left\{ \begin{array}{ll}
        \dfrac{1}{|x|},& \:\:\:  x \in (-\infty,\alpha)\cup (\alpha,\infty),
        \\
        \dfrac{1}{\alpha},& \:\:\: x \in [-\alpha,\alpha],
        \end{array}
        \right.
    \end{align*}
donde $\alpha>0$ es un valor dado.

\vspace{0.04\textheight}

Condiciones de Kato-Rellich $\implies$ el operador de Schrödinger $-\hslash^2 \Delta/2m + V$ es auto-adjunto $\implies$ ¿Dónde?
%\begin{teo}
%    Sea $\mathcal{V}:\field{R}\longrightarrow\field{R}$ una función integrable que puede expresarse como la suma de dos funciones reales $V_{1}$ y $V_{2}$, con $V_{1}\in L^{2}(\field{R},\field{C})$ y $V_{2}\in L^{\infty}(\field{R},\field{C})$. Entonces el operador de Schrödinger $-\hslash^2 \Delta/2m + V$ es auto-adjunto en $Dom(\Delta)$.
%    \label{teo:SCOperator}
%\end{teo}

\vspace{0.04\textheight}
\end{frame}

\begin{frame}{Conclusiones}

\begin{itemize}
    \item 
    \LARGE{\textcolor{Ocean}{Enfoque de las Ecuaciones Diferenciales Parciales}
    \vspace{0.2\textheight}}
    \item \LARGE{\textcolor{Ocean}{Enfoque del Análisis Funcional}} 
\end{itemize}

\end{frame}

\begin{frame}{}
    \Huge{\centerline{\usebeamercolor[fg]{title}Gracias por su atención}}
\end{frame}

\end{document}
